\begin{frame}
  \frametitle{Principle of Operation}
  A BJT can operate in 4 different modes:
  \begin{itemize}
  \item \textbf{Normal Mode}: Emitter junction of n-p-n transistoris forward biased and collector is reverse biased. Generally, at ON state a transistor remains in the normal mode.
  \item \textbf{Saturation Mode}: Both junctions are forward biased. Transistor acts like a short circuit.
  \item \textbf{Cut-off Mode}: Both junctions are reverse biased and transistor acts like an open circuit.
  \item \textbf{Inverse Mode}: Emitter is reverse biased and collector is forward biased. In practice, transistors are not commonly used in this mode.
  \end{itemize}
\end{frame}

\begin{frame}
    The microwave signal makes the base emitter junction forward biased at the negative portion of the signal and as a result electrons from the emitter region diffuses into the base and reaches the collector by drift mechanism. In the other half of the input signal the transistor remains in the OFF state, thus a pulse of current flows through the load connected in the collector circuit.
\end{frame}