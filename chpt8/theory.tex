\begin{frame}
  \frametitle{Theory}
  \framesubtitle{Definitions}
  $ e\phi_m$: work function of a metal, is the energy required for removal of an electron in metal from its Fermi-level to the outside vacuum.

  $e\phi_s$: work function of a semiconductor, is the energy required for removal of an electron in semiconductor from its Fermi-level to the outside vacuum.

  $e\chi$: electron affinity, is the energy required to lift an electron from the lowest level of conduction band to vacuum level.
\end{frame}

\begin{frame}
  \begin{figure}[h]
    \caption{Metal-semiconductor band diagrams before contact}
    \centering
    \includegraphics[width=0.4\textwidth]{./images/chapter8/fig8_1.png}
    \label{fig:8_1}
  \end{figure}

  After contact in figure \ref{fig:8_1}, the electron energy in the semiconductor must be lowered relative to the electron energy in the metal. This has the effect of creating a thin sheet of negative charge on metal side of junction and a depletion region of width \textit{w} on the semiconductor side consisting of immobile positive donor ions as shown in fig \ref{fig:8_2}
\end{frame}

\begin{frame}
  \begin{columns}
    \column{0.5\textwidth}
    \begin{figure}[h]
      \caption{Schottky barrier is a depletion layer formed at the junction of a metal and a semiconductor}
      \centering
      \includegraphics[width=0.7\textwidth]{./images/chapter8/fig8_2.png}
      \label{fig:8_2}
    \end{figure}

    \column{0.8\textwidth}
    \begin{figure}[h]
      \caption{Schottky barrier energy band diagram after contact at equilibrium}
      \centering
      \includegraphics[width=0.4\textwidth]{./images/chapter8/fig8_3.png}
      \label{fig:8_3}
    \end{figure}
  \end{columns}

  In figure \ref{fig:8_3}, the $e\phi_b$ is called the \emph{barrier height}.
  $e\phi_b = e\phi_m - e\chi$
  The voltage drop across the depletion layer \emph{w} of the semiconductor is called the \emph{equilibrium contact potential} $V_{bi}$. It prevents further diffusion of electrons from the conduction band of the semiconductor to the metal.
\end{frame}

\begin{frame}
  The current flowing through a schottky barrier diode is caused by:
  \begin{itemize}
  \item $I_o$ due to electron emission from metal to semiconductor junction over the barrier $(\phi_m - \chi)$
  \item Electron flow from semiconductor conduction band over the contact potential into the metal.
  \end{itemize}

  It can be shown that:
  \[ \phi_b \propto \exp \left(- \frac{e\phi_b}{RT}\right) \]
  \[ \therefore I_0 \propto \exp \left(- \frac{e\phi_b}{RT}\right) \]
\end{frame}

\begin{frame}
  For flow of current from semiconductor to metal, it is to be noted that the contact potential difference is reduced to $V_{bi} - V$ for forward bias and is increased to $V_{bi} + V$ for reverse bias. \emph{In forward bias, there is larger electron flow from semiconductor to metal.}
  The diode equation is similar to the diode equation for p-n junction which has the form:
  \[ I = A \exp \left(\frac{-e\phi_b}{kT}\right) \left( \exp \frac{-eV}{kT} -1 \right) \]
\end{frame}

\begin{frame}
  In Schottky Barrier Diodes, the forward current flows due to majority carriers from semiconductor to metal and there is no minority carrier current injection. Hence the phenomena of storage and recombination of minority carriers along with associated time delays do not occur in rectifying Schottky Barrier Diodes (unlike p-n junction rectifiers). Thus very fast switching can take place and the devices are very suitable for detection of high frequency signals.
\end{frame}
